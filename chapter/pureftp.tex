\chapter{PureFTP}\label{chapter:FTP}

\section{安装配置}

在Debian中,PureFTP通过/etc/pure-ftpd/conf/目录中的配置文件来确定启动参数。%
也可以通过查询启动参数的意义,直接采用命令启动,不过最好还是配置这些文件,%
通过PureFTP的内置转换工具,自动将配置文件转换为启动参数。

参数的意义参见官方手册,这里通过脚本来完成配置。

\begin{quote}
\kaishu
\textbf{Tips:}官方手册中除了包含编译、使用等内容以外,还有一个常见问题列表,这个列表%
非常值得参考,例如如何设置给用户设置共享文件夹等。
\end{quote}

\begin{Verbatim}[]
文件:5.ftpSet.sh
#!/bin/bash

PATH=/sbin:/usr/sbin:/bin:/usr/bin:/usr/local/sbin:/usr/local/bin; export PATH

echo -e "\n安装 pureFTP"

apt-get install pure-ftpd pure-ftpd-common -y

#配置 pureFTP
echo -e "\n配置 pureFTP"
confDir=/etc/pure-ftpd/conf

#默认已禁止匿名登录
echo "yes" > $confDir/ChrootEveryone
echo "yes" > $confDir/UnixAuthentication
echo "yes" > $confDir/ProhibitDotFilesWrite
echo "yes" > $confDir/ProhibitDotFilesRead
echo "yes" > $confDir/DontResolve

echo "5"  > $confDir/MaxClientsPerIP
echo "20" > $confDir/MaxIdleTime
echo "98" > $confDir/MaxDiskUsage

echo "117 117" > $confDir/Umask

echo -e "\n重启 pure-ftpd 服务"
systemctl restart pure-ftpd
\end{Verbatim}

\section{设置共享目录}
在chroot开启的时候如何给FTP用户设置共享目录呢?%
这个问题官方的常见问题手册给出了回答。%
PureFTP源码编译时,可以选择虚拟chroot选项,这种情形下,可以通过%
创建软链接:ln -s 来实现共享(注意使用绝对路径)。但是,Debian官方仓库中并未开启%
虚拟chroot选项。另一种更好的方法是,将共享目录挂载到用户目录下,命令为:%
mount -\,-bind。可写到/etc/fstab文件中,实现开机挂载。

\section{虚拟用户}
PureFTP的用户不一定非得是Linux用户,可以通过设定一个专门的FTP Linux账户,然后%
所有虚拟用户通过该专门的Linux账户访问FTP,具体命令参见官方手册。

增加虚拟、修改虚拟账户的命令是:pure-pw,其后跟的命令和Linux账户配置是一样的。%
存储虚拟账户的文件格式为:

\begin{Verbatim}[]
<account>:<password>:<uid>:<gid>:<gecos>:<home>:...
\end{Verbatim}

该文件包含了用户使用FTP的宽带等设置,其中,account、password、uid、gid、home为非空字段。
密码的加密方式和系统账户是一样的,home字段若以“\verb"/./"”结尾则表示chroot。%
添加虚拟用户后,Pure-FTP的命令可将该文件转化为二进制的PureDB,该文件是二进制文件,%
经过了排序处理,在有大量虚拟用户的情况下,可以加快处理速度。%
也可以使用内置命令直接复制系统账户。%

\begin{quote}\kaishu
\textbf{Tips:} 须查看Pure-FTP的配置文件,确认配置文件中默认设置的认证文件位置。%
例如,在Debian 7.x中,须做一下如下操作:\verb"cd /etc/pure-ftpd/auth",%
然后 \verb"ln -s ../conf/PureDB",否则虚拟账户会无法进行认证,可以查看log确认问题所在。
\end{quote}
