\chapter{邮件设置}

这里使用的软件为msmtp,该软件小巧且易于配置。邮箱客户端软件用的是mutt,该%
软件同样小巧,但是功能强大。

\begin{quote}\kaishu
\textbf{注意:}一般系统提供了mail、mutt、sendmail等工具,%
有时候系统提供的sendmail工具指向了其它的mail服务软件,例如在Debian 8.3中,sendmail是%
exim4这个软件的一个软链接,而mail命令来自于bsd-mailx。
\end{quote}

\section{mail配置}
mail会读取/etc/目录下的mail.rc文件,或者读取用户家目录下的.mailrc文件。但是%
mail似乎不能自定义配置文件的位置,在某些情况下会导致不便。可以在mail.rc文件中%
增加一行,使用msmtp发送邮件。

\begin{Verbatim}[]
set sendmail="/usr/bin/msmtp"
\end{Verbatim}

\section{msmtp配置文件}

msmtp会读取/etc/目录下的msmtprc文件,或者读取用户家目录下的.msmtprc文件,%
也可以用-C参数指定配置文件。一个示例如下:

\begin{Verbatim}[]
account default
host smtp.126.com                       <-- smtp服务器
port 25                                 <-- smtp服务器端口
from xxx@xxx.com
auth login
user xxx@xxx.com                        <-- 邮箱
password xxxx                           <-- 密码
logfile /root/Mail/Log/msmtp.log        <-- 日志位置
\end{Verbatim}


\section{mutt配置}
mutt会读取/etc/目录下的Muttrc文件,或读取用户家目录下的.muttrc文件,%
也可以用-F参数指定配置文件。一个示例如下:
\begin{Verbatim}[]
set from="xxx@xxx.com"                  <-- 发件人
set sendmail="/usr/bin/msmtp"           <-- 使用msmtp发送邮件
set use_from=yes
set realname="张三"
set editor="vim"                        <-- 发送邮件使用的文本编辑器
\end{Verbatim}

在使用nagios的邮件警报功能时,可以将配置文件放在一个特定的目录下,稍微修改一下%
上述参数即可。

\begin{Verbatim}[]
set from="xxx@xxx.com"
set sendmail="/usr/bin/msmtp -C /etc/nagios3/.msmtprc"
set use_from=yes
set realname="Nagios3 警报"
set editor="vim"
\end{Verbatim}

然后将发送邮件的命令改为:\verb"/usr/bin/mutt -F /etc/nagios3/.muttrc"。%
不过需要注意的是,要保证两份配置文件对nagios进程的拥有者是可读取的,例如%
默认情况下,Debian 8.x中nagios是以nagios用户和nagios组权限执行的。当然也可以%
将配置文件放置在nagios进程的拥有者的家目录底下,此时使用mail命令也可。
