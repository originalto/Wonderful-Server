\chapter{文件服务器}

文件服务器选定的系统版本是Debian 8.x amd64。

\section{设置本地源}
在离线环境下,为了安装软件方便,需要使用本地源,具体设置参见第\ref{chapter:mirror} 章。

\section{安装常用软件}
安装编译环境、Git、Vim、ctags、sudo 等软件包。
\begin{Verbatim}[]
文件:2.installPackages.sh
#!/bin/bash

echo -e "\nYong's 脚本: 安装 编译环境、Git、Vim、ctags、sudo 等软件包\n"


PATH=/sbin:/usr/sbin:/bin:/usr/bin:/usr/local/sbin:/usr/local/bin; export PATH

apt-get install build-essential vim -y
apt-get install ctags -y
apt-get install git git-gui -y
apt-get install sudo -y
\end{Verbatim}

\section{配置root使用环境}
主要设置bash环境、Vim环境等。
\begin{Verbatim}[]
文件:3.setRootEnv.sh
#!/bin/bash

PATH=/sbin:/usr/sbin:/bin:/usr/bin:/usr/local/sbin:/usr/local/bin; export PATH

echo -e "\nYong's 脚本: 设置 root 的用户环境\n"

homeDir=/root
chmod 750 $homeDir

# 创建 vim 配置文件目录
if [ ! -d "/root/.vim" ]
then
	mkdir $homeDir/.vim
fi

echo -e "Yong's 脚本: 复制 Vim 插件\n"

cp -av VimPlugin/* $homeDir/.vim/
cp -v Vim.dat $homeDir/.vimrc


echo -e "\nYong's 脚本: 在 /etc/profile 中设置 Intel 编译器的环境变量"

cat /etc/profile | grep -q "source /opt/intel"

if [ $? -ne 0 ]
then
	/bin/bash ./backupFile.sh /etc/profile
	cat profile.dat >> /etc/profile
fi

echo -e "\nYong's 脚本: 设置 .bashrc"

cat $homeDir/.bashrc | grep -q ".mybashset"

if [ $? -ne 0 ]
then
	cp -v rootBash.dat $homeDir/.mybashset
	/bin/bash ./backupFile.sh $homeDir/.bashrc
	echo "source $homeDir/.mybashset" >> $homeDir/.bashrc
fi
\end{Verbatim}

脚本中调用的其他文件是,

\begin{Verbatim}[]
文件:Vim.dat
"显示行号
set number

"显示光标当前位置
set ruler

"总是显示状态栏
set laststatus=2

" 高亮显示当前行/列
"set cursorline
"set cursorcolumn

"高亮显示搜索结果
set hlsearch

"查找前预览
set incsearch

"搜索时大小写不敏感
set ignorecase

"关闭兼容模式
set nocompatible

"禁止光标闪烁,gVim有效
"set gcr=a:block-blinkon0

"缩进方式
set tabstop=4
set softtabstop=4
set shiftwidth=4

"基于缩进或语法进行代码折叠
"set foldmethod=indent
set foldmethod=syntax
"启动 vim 时关闭折叠代码
set nofoldenable

"配色方案
colorscheme koehler

" 开启文件类型侦测
filetype on

" 根据侦测到的不同类型加载对应的插件
filetype plugin on


"语法高亮
if &t_Co > 2 || has("gui_running")
	syntax on
	set hlsearch
endif

"插件设置
map <F4> :NERDTreeToggle<CR>
map <F2> :TlistToggle<CR>
let Tlist_Use_Right_Window=1

nmap <C-H> <C-W>h " control+h进入左边的窗口
nmap <C-J> <C-W>j " control+j进入下边的窗口
nmap <C-K> <C-W>k " control+k进入上边的窗口
nmap <C-L> <C-W>l " control+l进入右边的窗口

"对齐插件设置,不是特别灵敏
let mapleader=";"
nmap <Leader>t& :Tab /&<CR>
vmap <Leader>t& :Tab /&<CR>
nmap <Leader>t= :Tab /=<CR>
vmap <Leader>t= :Tab /=<CR>
nmap <Leader>t" :Tab /"<CR>
vmap <Leader>t" :Tab /"<CR>
nmap <Leader>t: :Tab /:<CR>
vmap <Leader>t: :Tab /:<CR>
nmap <Leader>t:: :Tab /:\zs<CR>
vmap <Leader>t:: :Tab /:\zs<CR>
nmap <Leader>t, :Tab /,<CR>
vmap <Leader>t, :Tab /,<CR>
nmap <Leader>t,, :Tab /,\zs<CR>
vmap <Leader>t,, :Tab /,\zs<CR>
nmap <Leader>t<Bar> :Tab /<Bar><CR>
vmap <Leader>t<Bar> :Tab /<Bar><CR>
\end{Verbatim}

包含Intel编译器的使用环境。
\begin{Verbatim}[]
文件:profile.dat
source /opt/intel/bin/compilervars.sh intel64
source /opt/intel/impi/5.0.1.035/bin64/mpivars.sh
source /opt/intel/mkl/bin/mklvars.sh intel64 ilp64
\end{Verbatim}

root的Bash环境。
\begin{Verbatim}[]
文件:rootBash.dat
umask 0007

export LS_OPTIONS='--color=auto'
eval "`dircolors`"
alias ls='ls $LS_OPTIONS'
alias ll='ls $LS_OPTIONS -l'
alias l='ls $LS_OPTIONS -lA'

export GIT_EDITOR=vim
\end{Verbatim}


\section{增加用户}

增加用户、设置用户使用环境等。

\begin{Verbatim}[]
文件:4.addUsers.sh
#!/bin/bash

PATH=/sbin:/usr/sbin:/bin:/usr/bin:/usr/local/sbin:/usr/local/bin; export PATH

# 增加用户组
teamName=ZhangShuHaiTeam
groupadd $teamName

# 用户
accounts=$(cat accounts.dat)

# HOME目录
homeDir=/home

# 创建共享目录
shareDir=/home/TeamShare
mkdir $shareDir
chown root:$teamName $shareDir
chmod 2770 $shareDir

/bin/bash ./backupFile.sh /etc/fstab

for data in $accounts
do
	userName=$(echo $data | cut -d ":" -f 1)
	initPasswd=$(echo $data | cut -d ":" -f 2)

	#创建用户
	echo "Add User:" $userName
	useradd -G $teamName -m -s /bin/bash $userName
	echo $userName:$initPasswd | chpasswd

	#修改主文件目录权限
	if [ -d $homeDir/$userName ]
	then
		echo "Set HOMEDIR 750:" $homeDir/$userName
		chmod 750 $homeDir/$userName
	fi

	/bin/bash ./4.1.setuserEnv.sh $userName
	/bin/bash ./4.2.shareDir.sh   $userName
done

mount -a

/bin/bash ./4.3.setACL.sh

echo -e "\nAll Done!"
\end{Verbatim}

其中,accounts.dat每一行格式是:“用户名:密码”。

设置用户的Bash、Vim环境等。
\begin{Verbatim}[]
文件:4.1.setuserEnv.sh
#!/bin/bash

PATH=/sbin:/usr/sbin:/bin:/usr/bin:/usr/local/sbin:/usr/local/bin; export PATH

echo -e "\nYong's 脚本: 设置 $1 的用户环境\n"

userName=$1
homeDir=/home/$userName

# 创建 Vim 插件目录
if [ ! -d "$homeDir/.vim" ]
then
	mkdir $homeDir/.vim
fi

# 复制 Vim 插件
echo -e "Yong's 脚本: 复制 Vim 插件\n"

cp -av VimPlugin/* $homeDir/.vim/
chown -R $userName:$userName $homeDir/.vim

# 复制 Vim 配置文件
cp -v Vim.dat $homeDir/.vimrc
chown $userName:$userName $homeDir/.vimrc

# 复制 bash 配置文件
echo -e "\nYong's 脚本: 设置 .bashrc"

/bin/bash ./backupFile.sh $homeDir/.bashrc
cp -v Bash.dat $homeDir/.mybashset
chown $userName:$userName $homeDir/.mybashset

cat $homeDir/.bashrc | grep -q ".mybashset"

if [ $? -ne 0 ]
then
	echo "source $homeDir/.mybashset" >> $homeDir/.bashrc
fi

# 禁止图形界面登陆
/bin/bash ./backupFile.sh /etc/pam.d/gdm-password

echo "auth    required    pam_succeed_if.so user != $1 quiet_success" >> /etc/pam.d/gdm-password
\end{Verbatim}


\begin{Verbatim}[]
文件:bash.dat
umask 0007

export LS_OPTIONS='--color=auto'
eval "`dircolors`"
alias ls='ls $LS_OPTIONS'
alias ll='ls $LS_OPTIONS -l'
alias l='ls $LS_OPTIONS -lA'

export GIT_EDITOR=vim
\end{Verbatim}

FTP的共享文件夹设置。
\begin{Verbatim}[]
文件:4.2.shareDir.sh
#!/bin/bash

PATH=/sbin:/usr/sbin:/bin:/usr/bin:/usr/local/sbin:/usr/local/bin; export PATH

echo -e "\nYong's 脚本: 设置 $1 的用户环境\n"

useName=$1
homeDir=/home/$useName

shareDir=$homeDir/TeamShare

# 创建共享目录
if [ ! -d "$shareDir" ]
then
	mkdir $shareDir
	chown $useName:$useName $shareDir
else
	echo -e "\nYong's 脚本: $shareDir 目录已存在\n"
fi

# 开机挂载
echo "/home/TeamShare	$shareDir	none	defaults,bind	0	0" \
	>> /etc/fstab
\end{Verbatim}

某个特殊用户的ACL权限设置。
\begin{Verbatim}[]
文件:4.3.setACL.sh
#!/bin/bash

PATH=/sbin:/usr/sbin:/bin:/usr/bin:/usr/local/sbin:/usr/local/bin; export PATH

# 用户
accounts=$(cat accounts.dat)

# HOME目录
homeDir=/home

superUser=特殊用户名

for data in $accounts
do
	userName=$(echo $data | cut -d ":" -f 1)

	if [ $userName != $superUser ]
	then
		setfacl -m u:$superUser:rx	$homeDir/$userName	
	fi

done

echo -e "\nYong's 脚本: ACL权限设置完毕\n"
\end{Verbatim}

\section{FTP}
在对比了几种FTP之后,选择PureFTP,安装配置见第\ref{chapter:FTP} 章。


\section{防火墙}
配置防火墙,并设置开机启动。
\begin{Verbatim}[]
文件:6.setFirewallService.sh
#!/bin/bash

PATH=/sbin:/usr/sbin:/bin:/usr/bin:/usr/local/sbin:/usr/local/bin; export PATH

basepath=/root/FirewallScript # 脚本放置目录

if [ -f "$basepath" ]; then
	echo $basepath " 是一个文件,与要创建的文件夹重名"
	exit 1
fi

if [ ! -d "$basepath" ]; then
	mkdir -p $basepath
fi

cp -p FirewallScript/* $basepath

cat $basepath/Firewall.service.dat > /etc/systemd/system/Firewall.service

systemctl enable Firewall
systemctl start Firewall
\end{Verbatim}

配置iptables的脚本。
\begin{Verbatim}[]
文件:FirewallScript/Firewall.sh
#!/bin/bash

# 请先输入您的相关参数,不要输入错误了!
EXTIF="eth0" # 这个是可以连上 Public IP 的网络接口
INIF=""      # 内部 LAN 的连接接口;若无则写成 INIF=""
#INNET="192.168.100.0/24" # 若无内部网域接口,请填写成 INNET=""
INNET=""				  # 若无内部网域接口,请填写成 INNET=""
export EXTIF INIF INNET


# 1. 清除规则、设定默认政策及开放 lo 与相关的设定值
PATH=/sbin:/usr/sbin:/bin:/usr/bin:/usr/local/sbin:/usr/local/bin; export PATH
iptables -F
iptables -X
iptables -Z
iptables -P INPUT   DROP	# 默认规则设为拒绝,清除设置的时候要小心
iptables -P OUTPUT  ACCEPT
iptables -P FORWARD ACCEPT
iptables -A INPUT -i lo -j ACCEPT
iptables -A INPUT -m state --state RELATED,ESTABLISHED -j ACCEPT

basepath=/root/FirewallScript # 脚本放置目录

# 2. 启动额外的防火墙 script 模块
if [ -f "$basepath/deny.sh" ]; then
	/bin/bash $basepath/deny.sh	# 阻止某些网络进入
fi
if [ -f "$basepath/allow.sh" ]; then
	/bin/bash $basepath/allow.sh  # 运行某些网络进入
fi

# 3. 允许某些类型的 ICMP 封包进入
# AICMP="0 3 3/4 4 11 12 14 16 18"
# for tyicmp in $AICMP
# do
# 	iptables -A INPUT -i $EXTIF -p icmp --icmp-type $tyicmp -j ACCEPT
# done

# 4. 允许某些服务的进入,请依照你自己的环境开启
# iptables -A INPUT -p TCP -i $EXTIF --dport  21 --sport 1024:65534 -j ACCEPT # FTP
# iptables -A INPUT -p TCP -i $EXTIF --dport  22 --sport 1024:65534 -j ACCEPT # SSH
# iptables -A INPUT -p TCP -i $EXTIF --dport  25 --sport 1024:65534 -j ACCEPT # SMTP
# iptables -A INPUT -p UDP -i $EXTIF --dport  53 --sport 1024:65534 -j ACCEPT # DNS
# iptables -A INPUT -p TCP -i $EXTIF --dport  53 --sport 1024:65534 -j ACCEPT # DNS
# iptables -A INPUT -p TCP -i $EXTIF --dport  80 --sport 1024:65534 -j ACCEPT # WWW
# iptables -A INPUT -p TCP -i $EXTIF --dport 110 --sport 1024:65534 -j ACCEPT # POP3
# iptables -A INPUT -p TCP -i $EXTIF --dport 443 --sport 1024:65534 -j ACCEPT # HTTPS


# 5. 最终将这些功能储存下来吧!
iptables-save
\end{Verbatim}

设置允许通过防火墙的IP。
\begin{Verbatim}[]
文件:FirewallScript/allow.sh
#!/bin/bash
basepath=/root/FirewallScript # 脚本放置目录

ipAllow=$(cat $basepath/ipAllow.dat)

for userIP in $ipAllow
do
	iptables -A INPUT -i $EXTIF -s $userIP -j ACCEPT
	echo $userIP
done
\end{Verbatim}

设置阻止通过防火墙的IP。
\begin{Verbatim}[]
文件:FirewallScript/deny.sh
#!/bin/bash
basepath=/root/FirewallScript # 脚本放置目录

ipDeny=$(cat $basepath/ipDeny.dat)

for userIP in $ipDeny
do
	iptables -A INPUT -i $EXTIF -s $userIP -j DROP
done
\end{Verbatim}

其中,ipAllow.dat和ipDeny.dat分别包含了允许通过的IP和阻止通过的IP。

\begin{quote}
\kaishu
\textbf{Tips:}在清空防火墙的时候,需要注意事先将默认策略改为ACCEPT。命令为:%
iptables -P INPUT ACCEPT。
\end{quote}


\section{限制使用su和sudo}
设置部分用户可以使用某些系统命令。
\begin{Verbatim}[]
文件:7.suAndsudo.sh
#!/bin/bash

PATH=/sbin:/usr/sbin:/bin:/usr/bin:/usr/local/sbin:/usr/local/bin; export PATH

#限制使用部分sudo命令
/bin/bash ./backupFile.sh /etc/sudoers
cat sudoers.dat >> /etc/sudoers

#限制使用su
addgroup wheel
usermod -a -G wheel yong	# 加入wheel组的用户可以使用su命令

/bin/bash ./backupFile.sh /etc/pam.d/su
cat su.dat >> /etc/pam.d/su
\end{Verbatim}

该文件各项意义参见《鸟哥Linux私房菜——基础篇》。
\begin{Verbatim}[]
文件:sudoers.dat
%用户组名 ALL=/sbin/poweroff,/sbin/reboot
\end{Verbatim}

/etc/pam.d/su文件的配置可以参见该文件本身的注释说明,另外可参考《鸟哥Linux私房菜——基础篇》%
的“PAM模块简介”部分。
\begin{Verbatim}[]
文件:su.dat
auth	required	pam_wheel.so
\end{Verbatim}

\section{删除系统自带账户}
作为文件服务器,有一些系统账户用不到,可以删除。
\begin{Verbatim}[]
文件:8.delSysUserAndGroup.sh
#!/bin/bash

PATH=/sbin:/usr/sbin:/bin:/usr/bin:/usr/local/sbin:/usr/local/bin; export PATH

delUsers="lp sync news uucp games"
delGroups="lp news uucp games dip"

for userName in $delUsers
do
	echo "Delete User: " $userName
	userdel $userName
done

echo -e "Delete users done.\n"

for groupName in $delGroups
do
	echo "Delete Group: " $groupName
	groupdel $groupName
done

echo -e "Delete groups done.\n"
\end{Verbatim}

\section{温度监控}
CPU等温度监控,安装lm-sensors模块。
\begin{itemize*}
  \item 安装:apt-get install lm-sensors -y
  \item 探测:sensors-detect
  \item 查看:sensors
\end{itemize*}

硬盘温度监控,安装hddtemp模块。详见“有道笔记”中《hddtemp》。
\begin{itemize*}
  \item 安装:apt-get install hddtemp
  \item 探测:hddtemp /dev/sda /dev/sdb ... ...
\end{itemize*}
