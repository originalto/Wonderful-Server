\chapter{Intel Parallel XE Studio}

\section{安装XE}
这里选择Intel Parallel XE Studio Cluster Edtion,因为该版本包含了MPI库的内容。%
官网上可以申请一年使用的License,为了方便这里选择2015版本,虽然不是最新版本,%
但是省去了一年一申的麻烦。

查看安装帮助可知,安装可以使用静默方式,事先用虚拟机安装一遍,可以得到配置文件,%
安装命令为:./install.sh -d ./autoInstall.cfg。该文件内容为:
\begin{Verbatim}[]
文件:autoInstall.cfg
ACCEPT_EULA=accept
INSTALL_MODE=NONRPM
CONTINUE_WITH_OPTIONAL_ERROR=yes
PSET_INSTALL_DIR=/opt/intel
CONTINUE_WITH_INSTALLDIR_OVERWRITE=yes
PSET_MODE=install
ACTIVATION_LICENSE_FILE=/media/sf_share/XE2015/Crack/zwt.lic
ACTIVATION_TYPE=license_file
AMPLIFIER_SAMPLING_DRIVER_INSTALL_TYPE=build
AMPLIFIER_DRIVER_ACCESS_GROUP=vtune
AMPLIFIER_DRIVER_PERMISSIONS=666
AMPLIFIER_LOAD_DRIVER=yes
AMPLIFIER_C_COMPILER=/usr/bin//gcc
AMPLIFIER_KERNEL_SRC_DIR=/lib/modules/3.16.0-4-amd64/build
AMPLIFIER_MAKE_COMMAND=/usr/bin//make
AMPLIFIER_INSTALL_BOOT_SCRIPT=yes
AMPLIFIER_DRIVER_PER_USER_MODE=no
ENVIRONMENT_LD_SO_CONF=no
MPSS_RESTART_STACK=no
PHONEHOME_SEND_USAGE_DATA=no
COMPONENTS=;intel-mpi-rt__noarch;... ... 一系列组件
\end{Verbatim}

接下来就可以使用该文件进行静默安装,命令为:./install.sh -s autoInstall.cfg。%
这样就省去了以后安装需要交互操作的麻烦。

安装完毕后,还需要配置用户环境。将以下内容加入/etc/profile文件中即可。
\begin{Verbatim}[]
source /opt/intel/bin/compilervars.sh intel64
source /opt/intel/impi/5.0.1.035/bin64/mpivars.sh
source /opt/intel/mkl/bin/mklvars.sh intel64 ilp64
\end{Verbatim}

\section{Fortran测试}
接下来对一个简单的Fortran程序进行并行测试。
\begin{Verbatim}[]
文件:hello.f90
program main
use mpi
implicit none

	integer ::  id, err, comm, psize

	call MPI_INIT( err )
	call MPI_COMM_DUP( MPI_COMM_WORLD, comm, err )
	call MPI_COMM_RANK( comm, id, err )
	call MPI_COMM_SIZE( comm, psize, err )

	write(*,*) id, psize,  " hello"

	call MPI_BARRIER( comm, err )
	call MPI_FINALIZE( err )
end program
\end{Verbatim}

编译:
\begin{Verbatim}[]
mpiifort -o main hello.f90
\end{Verbatim}

执行:
\begin{Verbatim}[]
mpirun -np 4 ./main
\end{Verbatim}
\noindent 这里需要注意可执行文件的路径,即这里的“./”不要漏了。

运行结果:
\begin{Verbatim}[]
           3           4  hello
           0           4  hello
           2           4  hello
           1           4  hello
\end{Verbatim}
