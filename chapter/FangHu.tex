\chapter{安全防护}

\section{防暴力破解}

fail2ban 是 Linux 上的一个著名的入侵保护的开源框架,它会监控多个系统的日志文
件(例如:/var/log/auth.log或者/var/log/secure)并根据检测到的任何可疑的行
为自动触发不同的防御动作,可以用来防范如ssh、ftp等被暴力破解。

Debian 8.x安装fail2ban后,配置文件位置在/etc/fail2ban目录下:
\begin{Verbatim}[]
drwxr-xr-x 2 root root  4096 5月  30 21:54 action.d        <-- 定义动作
-rw-r--r-- 1 root root  1525 3月  15  2014 fail2ban.conf
drwxr-xr-x 2 root root  4096 3月  19  2014 fail2ban.d
drwxr-xr-x 2 root root  4096 5月  30 21:23 filter.d        <-- 定义日志分析策略
-rw-r--r-- 1 root root 13883 5月  30 21:48 jail.conf       <-- 设置需要保护的服务
drwxr-xr-x 2 root root  4096 3月  19  2014 jail.d
\end{Verbatim}

一般情况下,我们只需要修改jail.conf即可,或者在jail.d自定义。jail.conf文件最开头的%
[DEFAULT]部分是全局设置,可以在局部设置中覆盖全局变量。

\begin{Verbatim}[]
[DEFAULT]
ignoreip = 127.0.0.1/8
ignorecommand =
bantime  = 600                              #封锁时间,负数表示永久封锁
findtime = 600                              #在多长时间以内达到条件则开始执行封锁
maxretry = 3                                #如600秒达到3次则执行
backend = auto
usedns = warn
destemail = zhenyanxl@126.com               #警报邮件发给该地址
sendername = Fail2Ban
sender = "Fail2Ban 警报"
banaction = iptables-multiport
mta = sendmail                              #邮件参数,可改为mail等
protocol = tcp
chain = INPUT
action_ = %(banaction)s[略...]              #仅封锁
action_mw = %(banaction)s[略...]            #封锁+发送邮件
            %(mta)s-whois[略...]
action_mwl = %(banaction)s[略...]           #封锁+发送邮件(邮件包含日志信息)
             %(mta)s-whois-lines[略...]
action = %(action_)s                        #默认动作
\end{Verbatim}

例如,可以修改ssh和pure-ftp的默认设置,让封锁之后发送警报邮件。

\begin{Verbatim}[]
[ssh]
enabled  = true                              <-- 开启监控
port     = ssh
filter   = sshd
mta      = mail                              <-- 使用mail命令发送邮件
action   = %(action_mwl)s
logpath  = /var/log/auth.log                 <-- 日志位置
maxretry = 6

[pure-ftpd]
enabled  = true
port     = ftp,ftp-data,ftps,ftps-data
filter   = pure-ftpd
mta      = mail
action   = %(action_mwl)s
logpath  = /var/log/syslog
maxretry = 6
\end{Verbatim}

\section{防端口扫描}

在服务器没有使用硬件防火墙的情况下,为了防止有人恶意获取服务器用途等信息,防端口扫描%
十分必要。PortSentry是一款非常容易使用的防恶意扫描工具,它可设置如下几种抵挡方式:

\begin{itemize*}
  \item 路由表:通过设置路由表,将信息流导向虚假位置。
  \item TCP\_wrappers:将对方的IP写入/etc/hosts.deny文件。
  \item iptables:通过修改iptables,将对方数据包过滤。
  \item syslog():给出日志消息,返回警告信息。
  \item 自定义脚本。
\end{itemize*}

其中,路由表和iptables是两种较好的方法。

在Debian 8.x中,安装PortSentry以后其配置文件在/etc/portsentry目录中,主要的配置文件%
是portsentry.conf,默认内容如下:

\begin{Verbatim}[]
TCP_PORTS="1,11,15,79,111,119,143,540,635,1080,1524,2000,5742,6667,12345,12346,20034
UDP_PORTS="1,7,9,69,161,162,513,635,640,641,700,37444,34555,31335,32770,32771,32772,
ADVANCED_PORTS_TCP="1024"
ADVANCED_PORTS_UDP="1024"
ADVANCED_EXCLUDE_TCP="113,139"
ADVANCED_EXCLUDE_UDP="520,138,137,67"
IGNORE_FILE="/etc/portsentry/portsentry.ignore"
HISTORY_FILE="/var/lib/portsentry/portsentry.history"
BLOCKED_FILE="/var/lib/portsentry/portsentry.blocked"
RESOLVE_HOST = "0"
BLOCK_UDP="0"        <-- 默认不阻挡UDP扫描,将值改为 1
BLOCK_TCP="0"        <-- 默认不阻挡TCP扫描,将值改为 1
KILL_ROUTE="/sbin/route add -host $TARGET$ reject"       <-- 利用路由阻挡
#KILL_ROUTE="/sbin/route add $TARGET$ 333.444.555.666"   <-- 将数据流导向不存在的主机
KILL_HOSTS_DENY="ALL: $TARGET$ : DENY"   <-- 利用TCP_wrappers阻挡
SCAN_TRIGGER="0"
\end{Verbatim}


对UDP和TCP,PortSentry分别有三种检测模式:

\begin{itemize*}
  \item portsentry-tcp:TCP基本端口绑定模式;
  \item portsentry-udp:UDP基本端口绑定模式;
  \item portsentry-stcp:TCP秘密扫描检测模式;
  \item portsentry-sudp:UDP秘密扫描检测模式;
  \item portsentry-atcp:TCP高级秘密扫描检测模式;
  \item portsentry-audp:UDP高级秘密扫描检测模式。
\end{itemize*}

推荐使用高级秘密扫描检测模式,配置文件是/etc/default/portsentry,修改其默认值%
然后重启portsentry服务即可。

\begin{Verbatim}[]
TCP_MODE="atcp"
UDP_MODE="audp"
\end{Verbatim}
